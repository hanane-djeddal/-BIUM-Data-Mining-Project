% pour genree un pdf: faire
% pdflatex exemple.tex
\documentclass{article}

%% Paquets LateX utiles

\usepackage[utf8]{inputenc} 		% encodage des caracteres utilise (pour les caracteres accentues) -- non utilise ici.
%\usepackage[latin1]{inputenc} 		% autre encodage
\usepackage[french]{babel}		% pour une mise en forme "francaise"
\usepackage{amsmath,amssymb,amsthm}	% pour les maths
\usepackage{graphicx}			% pour inclure des graphiques

\usepackage[hidelinks]{hyperref}
\usepackage{color}			% pour ajouter des couleurs dans vos textes
\usepackage{geometry}
\geometry{hmargin=2.5cm,vmargin=3cm}
\renewcommand{\contentsname}{\centering Contents}


\begin{document}
\begin{titlepage}
    \begin{flushleft}
    \includegraphics[width=11em]{logo.png}\\[1.5cm]
    \end{flushleft}
    \begin{center}
        \textsc{{\LARGE \color{blue} Master Données, Apprentissage et Connaissances-DAC}}\\[5cm]
        \textsc{\LARGE{RAPPORT PROJET BIUM}}\\[1cm]
        \textsc{\vspace{10pt}\Huge{Un problème de fidelisation des clients en Télécommunication}}\\[4cm]
        \begin{minipage}{1\textwidth}
            \begin{flushleft} \large
            \textsc{\LARGE{Realisé par :}}\\[0.5cm]
            \textsc{Hanane Djeddal}\\
            \textsc{Liticia Touzari}\\[1.5 cm]
            \end{flushleft}
        \end{minipage}
        \vfill
    \end{center}
  \end{titlepage}
  

\tableofcontents					% si on veut une table des matieres


\newpage

\vspace*{\stretch{0.5}}
  \begin{center}
\section*{\LARGE{Introduction}}
  \end{center}
\Large{\paragraph{}
        Le Churn, qui designe le désabonnement des clients, est un problème très reccurent dans l'industrie.
        Trouver des nouveaux clients étant plus difficile, surtout si le marché est saturé, comme
        est le cas en télécommunication, plusieurs entreprises prefère investir dans 
        la prévention de perte de clients.
\paragraph{}
        Plusieurs solutions existent pour prédire le churn qui
        permettent de cibler les clients suspectible de se désabonner et de leur offrir, par la suite,
        des abonnements plus intéressants pour les convaincre à prolonger leurs contracts. Une autre 
        question se pose, est ce que c'est possible de prévenir les clients de vouloir se désabonner? 
\paragraph{}        
        Dans ce projet, on va utiliser le Data Mining sur un ensemble de bases de données des 
        opérateurs téléphiniques pour etudier les profiles des clients pour pouvoir identifier les
        facteurs clés de churn.
}
\vspace*{\stretch{1}}
\newpage

\section{Présentation de la Problématique}
\paragraph{}
    Dans ce projet, on souhaite étudier les profiles des clients afin d'identifier les facteurs clés
qui causent le Churn. Notre schéma doit nous permettre, donc, de répondre aux besoins suivants : \\
1- Savoir les tranches d'age de l'ensemble des clients. \\
2- Analyser : 
\begin{itemize}
    -Le nombre des clients qui se désabonnent.\\
    -Le nombre total de minutes par jour. \\
    -Le nombre des appels par jour.\\
    -Le nombre des sms par jour.\\
    -Le nombre de message vocal par jour.\\
    -Le nombre des appels le soir.\\
    -Le nombre des sms le soir.\\
    -Le nombre de message vocal le soir.\\
    -Le nombre des appels au service client.\\
    -La charge moyenne.
\end{itemize}
En fonction de :
\begin{itemize}
    -Sexe.\\ 
    -Tranche d'age.\\
    -État civil.\\
    -Ancienneté.\\ 
    -Type d'abonnement téléphonique.\\
    -Profession \\
    -Si le client a un ordinateur ou pas.
\end{itemize}  
On veut aussi : \\
3-Analyser l'ancienneté des clients par sexe, tranche d'age, état civil, type 
d'abonnement, profession et si le client a un oridinateur ou pas.\\
4- La durée moyenne de l'abonnement des clients qui se désabonnent.\\
5- Le taux de désabonnement par mois, saison et année.\\
\section{Schéma Conceptuel:}
À partir des besoins cités dans la partie précédante, on construit le schéma conceptuel suivant: \\
\includegraphics[width=11em]{logo.png}\\[1.5cm]
}



\end{document}


